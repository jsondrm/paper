\chapter{绪\hskip 0.4cm 论}
\label{chap1}



\begin{kaishu}
   本章叙述了软件可信的研究背景与意义、软件可信国内外研究现状,并介绍了本文的研究内容以及文章的目录结构\cite{2011基于“环境}。
\end{kaishu}
\section{研究背景与意义}

处于信息时代的软件行业飞速发展,软件应用环境也日益复杂,对软件质量的要求也越来越高,给用户提供一个可信的软件很有必要。
软件内部业务处理逻辑更加复杂使其整体规模相较之前日益庞大,相关软件人员必须投入更多精力确保软件的可信性。
% 随着软件系统规模逐渐庞大,软件内部复杂性不断提高,软件开发和设计人员对软件是否满足可信要求日益重视。
各种类型的软件已经渗透在人们的生活之中,网上购物软件、地图导航软件和医院挂号软件等。比如飞机地铁,一半以上的功能都是由软件控制的。还有其他关键应用领域(银行,证券交易,军事等),这些软件一旦运行发生错误,可能会造成灾难性的后果,损失巨额财产或失去生命\cite{2009贝叶斯网络在软件可信性评估指标体系中的应用}。

软件可信的要求之一就是软件能按照用户预期正常工作,即使在输入错误或者其他异常情况发生的条件下,软件也不会崩溃\cite{Wang2006Trustworthiness}。
因此,需要评估软件的可信性,当软件满足可信要求之后再投入使用,可以适当地降低软件出现问题的概率。
% 这就要求软件在投入使用之前进行可信性评估,确保软件的可信等级满足一定的等级要求。
用户对于刚推出软件的可信性不满意,是因为这些软件产品的构造与保障技术有待完善,使其存在许多已知或未知的缺陷。
% 由于可信软件构造与运行保障技术不够完善,使得软件产品在推出时就含有很多已知或未知的缺陷\cite{陈翔2016静态软件缺陷预测方法研究},不能很好地满足用户的软件可信性需求。
构造和开发软件,如果想保证其是可信的,必然要选择对其可信性进行计算并加以评估,以保障运行的安全可靠,这是当下软件技术发展的重要趋势\cite{熊伟2010基于}。
% 文献\cite{InverardiP1994AdaptationDependability}认为构造和开发可信软件,对其进行可信性计算和评估以保障其安全可靠运行已成为现代软件技术发展和应用的重要趋势和必然选择。
于是,国内外相关政府、科研机构和企业制定计划,把目光聚焦于可信软件研究。
例如,国家科技部“863”计划;
% 美国的《国家软件发展战略(2006-2015)》把开发可信软件放在首位。
2019年,华为与华东师范大学签约,双方深入开展合作,支持高校高可信创新实验室的建设。

城市轨道交通的信号系统逐渐采用基于通信的列车运行控制系统CBTC(Communication-Based Train Control System),联锁系统是其子系统,为了保证列车在线路上的安全运行,信号机、道岔和进路之间必须满足一定的制约关系,联锁系统的作用就是保证该关系的实现\cite{2006城市轨道交通信号设备}。
计算机联锁是地铁信号系统的安全核心,联锁软件是实现车站信号系统功能安全的核心软件模块,
可以间接通过提高地铁的自动化程度和管理水平,实现提高地铁整体的运营效率的目标,并且行车指挥调度人员的工作强度明显降低\cite{彭涛2014基于SCADE的信息物理融合系统的分析和设计方法}。
% 对提高地铁运营效率、自动化程度、管理水平以及减少行车指挥调度人员的工作强度具有最直接的影响。
对联锁软件进行可信评估可为现有的软件开发过程提供参考和指导,改进之前做的不好的地方,提高软件整体的可信度,满足城市轨道交通安全、高效运行的要求。

高可信的软件不仅降低企业的维护成本还能提升使用人员的好感度。
文献\cite{杨学伟2012论城市地铁文化建设的策略}提到随着许多城市地铁建设步伐的不断加快和工作人员日益增加,大多数人上下班首选的交通工具从公交转变成了地铁,
% 知网没有这篇文献
% 文献\cite{邹莉2018对地铁通信系统常见故障问题的分析和若干建议}提到随着地铁建设步伐的不断加快,许多城市地铁已经成为人们上下班首选的交通工具,
地铁一旦运行发生问题,短短几分钟就可能导致交通拥堵或者瘫痪。为了保障其投入使用之后能正常稳定地运行,有必要对这个安全有关领域的软件系统进行可信评估,发现软件存在的问题并改进,尽量避免运行时发生故障,带给乘客良好的用户体验,保证乘客生命安全。

\section{国内外研究现状}
可信软件的构造要以理论为基础,为此不同专家对于软件可信分别选择不同的切入点展开探讨。
%为了给如何构造可信软件提供理论依据,国内外专家对于软件可信分别选择不同的切入点展开探讨。
% 为了给构造可信软件提供理论依据,国内外学者从不同角度对软件可信进行研究。
文献\cite{蔡斯博2010一种支持软件资源可信评估的框架}通过证据收集来动态构造软件可信指标体系并应用于北京大学软件资源库。
文献\cite{杨善林2009一种基于效用和证据理论的可信软件评估方法}以工业领域中的金属液态检测软件为例,运行指标树生成算法得到软件可信指标树来动态构造软件可信指标系统。
文献\cite{赵倩2011基于}提到\cite{丁学雷2010面向验证的软件可信证据与可信评估}在分析了传统可信证据收集与分类方法的不足之后,提出了一个基于验证的可信证据模型。
% 文献\cite{丁学雷2010面向验证的软件可信证据与可信评估}分析了传统可信证据收集与分类方法的不足,提出了一个基于验证的可信证据模型。
文献\cite{Zhang2012An}对某个航空航天软件的可信性采用了传统的模糊综合评估模型进行评估。
% 文献\cite{Zhang2012An}使用传统的模糊综合评估模型对软件可信性进行评估,并对某个航空航天软件进行了实验。
文献\cite{王婧2015航天嵌入式软件可信性度量方法及应用研究}将可信评估应用在航天领域上,提出了基于出厂报告的可信度量模型与评估体系。
文献\cite{AmorosoE1991TowardanApproach,AmorosoE1994Aprocess-orientedmethodology}聚焦于软件开发过程提出了一种称为可信软件方法学TSM(Trusted Software Methodology)的理论,其制定了软件开发的安全原则和软件工程原则;然后,对软件进行可信性度量的原则是以可信规则为准,对比软件开发方法是否与其保持一致。
% 然后,通过确定软件开发方法是否与可信规则一致来对软件可信性进行度量。
\cite{Abrial2013Modeling}阿布瑞尔教授创立了可信软件开发“B方法”,并应用于航空航天、轨道交通和汽车电子等领域。
文献\cite{伍志强2019基于可信证据的软件可信性计算模型设计与工具实现}针对核装备领域,提出了基于可信证据的软件可信性计算模型。
文献\cite{李岩2017软件可信性静态度量模型设计与工具实现}基于软件源代码提出了一种静态可信度量体系。
文献\cite{于本海2014可信软件测度理论与方法}建立了软件过程与软件产品可信结构方程模型,更关注过程实体,过程行为,过程产品。

目前在轨道交通领域,关于可信评估的研究比较匮乏。
% 目前几乎没有面向轨道交通领域的可信评估相关研究。
对于可信评估来说,权重的获取是至关重要的一环。许多研究确定权重主要采用专家经验获取、主成分分析法、层次分析法、信息熵、粗糙集理论等方法\cite{张俊2016一种基于软件属性相互影响和重要性的属性权重配方法}。在本文提出的轨道交通联锁软件可信量化评估模型中,首先建立软件面向全生命周期的可信评估指标体系,然后分别构建可信量化评估结构方程模型以确定指标权重;接着提出了指标改进优先指数和改进优先级概念并设计了基于指标改进优先指数和单位贡献成本的可信性分配算法;最后,开发了基于软件全生命周期的轨道交通联锁软件可信量化评估工具。旨在依据当前联锁软件评估结果,结合联锁的自身特点,提高联锁软件的可信性,为联锁软件质量的提高和项目的管理提供参考意见。

\section{研究内容}
轨道交通作为安全攸关领域之一,针对其遵循的国际标准技术规范以及联锁软件的特点,
本文本文选取了四个属性,分别为功能性、可靠性、安全性和可维护性,建立软件全生命周期可信量化评估模型。
% 本文选取了四个属性,分别为功能性、可靠性、安全性和可维护性,建立软件全生命周期可信量化评估模型。
由于联锁软件对指标的重视程度不同,提出改进优先指数的概念\cite{2011改进的最大优先指标及在计算机化自适应诊断测验中的应用},并在此基础上设计两种软件可信性分配算法。最后,开发联锁软件的可信量化评估工具。

\begin{itemize}
	\item 建立轨道交通联锁软件可信量化评估模型。首先构造面向全生命周期的阶段可信评估指标体系,每个阶段评估目标是阶段的可信度量值,一级指标为每个阶段的属性,二级指标为每个阶段的度量元。然后建立可信评估结构方程模型,对其标准化路径图中的因素负荷量进行归一化得到各级指标的权重,自下而上计算软件的可信度量值并进行可信等级的划分。
	
	\item 设计两种软件可信性分配算法。可信性分配是为了提升软件的可信等级,将子属性待提高可信值分配给其下一级的度量元。其一法根据指标改进优先指数进行可信性分配,改进优先指数是综合考虑了该指标的权重和当前的可信值。另一种分配算法基于贪心选择策略,按照度量元的单位贡献成本,从小到大依次选取,得到最低改进成本。
	
	\item 开发轨道交通联锁软件可信量化评估工具。主要功能为三大模块:一是基本信息管理模块,包含产品信息维护和人员信息维护;二是全生命周期可信评估模块,由需求分析、系统设计、编码实现和软件测试阶段可信评估四个子模块构成,每个子模块功能有可信证据输入、权重分配和可信值计算;三是评估结果模块含有数据可视化子模块与评估报告子模块,数据可视化子模块对每个阶段属性和子属性的可信值分布以雷达图和堆叠条形图的形式进行可视化展示,评估报告子模块给出软件等级等信息和具体改进意见。
	
\end{itemize}
\section{文章结构}
第二章中首先介绍了软件可信的相关理论;然后介绍了结构方程模型基本概念;最后介绍工具实现使用的语言和框架。

第三章中介绍了轨道交通联锁软件的可信量化评估模型。首先建立每个阶段的可信评估指标体系和可信评估的结构方程模型。然后对数据进行效度和信度检验,将路径图和数据输入AMOS软件,得到标准化的路径系数。最后,对路径系数归一化确定每个指标的权重,自下而上计算软件的可信值,根据可信量化分级模型表,划分软件的可信等级。

第四章主要介绍了两种可信性分配算法。首先在同时考虑指标权重和可信值的前提下,提出指标改进优先指数的概念。轨道交通联锁软件对每个指标的侧重程度不同,提升软件可信等级时,可选择按照度量元的改进优先指数对子属性可信值进行分配。由于在实际的项目中,可能会受到成本约束,因此,在此基础上又提出度量元单位贡献成本的概念,基于贪心选择策略的分配算法,得到软件可信等级提高到某一级别时的最低改进成本。该算法得出的具体分配方案可供后续的开发过程参考,具有指导意义。

第五章主要介绍了轨道交通联锁软件可信评估系统。该系统第一个模块是基本信息管理,包括产品信息维护和人员信息维护;第二个模块是软件全生命周期四个阶段的可信评估,每个阶段包括可信证据输入,权重分配和可信值计算;最后的评估结果输出模块通过可信值图谱、堆叠区域图和表格对评估完的数据进行可视化展示,评估报告可供决策人员查看下载,安排后续的相关改进计划。

第六章对文章内容进行总结,并提出未来的研究设想。





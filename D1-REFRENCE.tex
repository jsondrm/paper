%% google scholar GB/T 7714

\begin{thebibliography}{2}
\setlength{\baselineskip}{25pt}

\bibitem{AD}
Inverardi P , Tivoli M . \emph{The Future of Software: Adaptation and Dependability[C]//}, Software Engineering, International Summer Schools, Issse, Salerno, Italy, Revised Tutorial Lectures. DBLP, 2006.

\bibitem{caisibo}
蔡斯博, 邹艳珍, 邵凌霜, et al. \emph{一种支持软件资源可信评估的框架[J]}. 软件学报, 2010, 21(2):359-372.

\bibitem{yangshanlin}
杨善林[1], 丁帅[1], 褚伟[1]. \emph{一种基于效用和证据理论的可信软件评估方法[J]}. 计算机研究与发展, 2009(7).

\bibitem{dingxuelei}
丁学雷, 王怀民, 王元元, et al. \emph{面向验证的软件可信证据与可信评估[J]}. 计算机科学与探索, 2010, 4(1):46-53.

\bibitem{FCEM}
Zhang Y , Zhang Y , Hai M . \emph{An Evaluation Model of Software Trustworthiness Based on Fuzzy Comprehensive Evaluation Method[C]//}. International Conference on Industrial Control \& Electronics Engineering. IEEE Computer Society, 2012.

\bibitem{TAMST}
Amoroso E , Nguyen T , Weiss J , et al. \emph{Toward an Approach to Measuring Software Trust.[C]}// IEEE Computer Society Symposium on Research in Security \& Privacy. IEEE, 1991.

\bibitem{pomist}
Amoroso E , Taylor C , Watson J , et al. \emph{A process-oriented methodology for assessing and improving software trustworthiness[C]}// Conference on Ccs. DBLP, 1994.


\bibitem{TIC}
Schneider F B . \emph{Trust in Cyberspace[J]}. 1999.

\bibitem{liuke}
刘克, 单志广, 王戟, et al. \emph{“可信软件基础研究”重大研究计划综述[J]}. 中国科学基金, 2008, 22(3):145-151.


\bibitem{chenhuowang}
陈火旺, 王戟, 董威. \emph{高可信软件工程技术[J]}. 电子学报, 2003, 31(S1):1933-1938.

\bibitem{DBCT}
Becker S . \emph{Trustworthy software systems: a discussion of basic concepts and terminology.[J]}. Acm Sigsoft Software Engineering Notes, 2006, 31(6):1-18.

\bibitem{QRMBTA}
Tao H , Chen Y . \emph{A QUANTITATIVE RELATION MODEL BETWEEN TRUSTWORTHY ATTRIBUTES[C]}// Quantitative Logic \& Soft Computing-the Ql \& sc. 2014.


\bibitem{ISO2}
ISO , International Organization for Standardization , ISO 9126-1:2001 , Software engineering - Product quality, Part 2:. External metrics , 2001.

\bibitem{ISO3}
ISO , International Organization for Standardization , ISO 9126-1:2001 , Software engineering - Product quality, Part 3 . Internal m etrics , 2001.

\bibitem{guliang}
古亮, 郭耀, 王华, et al. \emph{基于TPM的运行时软件可信证据收集机制[J]}. 软件学报, 2010, 21(2):373-387.

\bibitem{wangjing}
王婧, 陈仪香, 顾斌, et al. \emph{航天嵌入式软件可信性度量方法及应用研究[J]}. 中国科学:技术科学, 2015(2):221-228.

\bibitem{wuzhiqiang}
伍志强 \emph{基于可信证据的软件可信性计算模型设计与工具实现}

\bibitem{wuminglong}
吴明隆. \emph{结构方程模型——AMOS的操作与应用[M]}. 重庆:重庆大学出版社,2010.

\bibitem{wangkai}
王凯. \emph{一种新的评价结构方程模型拟合效果的校正拟合指数[D]}. 2018.

\bibitem{zhaoyiqing}
赵怡晴[1, 2], 胡晓运[1, et al. \emph{产品质量风险监控绩效评估的结构方程模型[J]}. 数理统计与管理, 2016, 35(5):856-867.

\bibitem{zhoutao}
周涛, 鲁耀斌, ZHOUTao, et al. \emph{结构方程模型及其在实证分析中的应用[J]}. 工业工程与管理, 2006(5).

\bibitem{EN50126}
铁路应用-可靠性,可用性,可维护性和安全性(RAMS)的规范和示例.

\bibitem{taohongwei}
陶红伟. \emph{基于属性的软件可信性度量模型研究}. 华东师范大学博士论文. 2011.

\bibitem{zhangweixiang}
张卫祥, 刘文红, 吴欣. \emph{软件可信性定量评估:模型、方法与实施[M]}. 清华大学出版社, 2015.

\bibitem{duyuanwei}
杜元伟. \emph{风险投资项目组合优化方法与其应用研究[D]}. 长春:吉林大学,2007.

\bibitem{wangxueli}
王雪丽, 张璇, 李彤, et al. \emph{面向可信软件的风险评估方法[J]}. 计算机工程与应用唯一官方网站, 2016, 52(19):97-101.

% 分界线

% \bibitem{SAGIWSN2014}
% M. Maamar, J. Liu and W. Liu, \emph{A new lightweight link quality based reputation model for Space-Air-Ground Integrated Wireless Sensor Network (SAGIWSN)}, Electronics, Computer and Applications, 2014 IEEE Workshop on. IEEE, 2014: 230-236.

% \bibitem{GIG}
% P. Wolfowitz, \emph{Global Information Grid (GIG) Overarching Policy}, US Department of Defense, directive, 2002 (8100.1).

% \bibitem{shenrongjun}
% R. Shen, \emph{Some Thoughts of (Chinaese) Integrated Space-Ground Network System}, Engineering Science, 2006, 10: 003 (in Chinese).

% \bibitem{SAG2013}
% H. Liu, J. Zhang and L. L. Cheng, \emph{Application examples of the network fixed point theory for space-air-ground integrated communication network}, Ultra Modern Telecommunications and Control Systems and Workshops (ICUMT), 2010 International Congress on. IEEE, 2010: 989-993.

% \bibitem{SCR}
% K. L. Heninger, \emph{Specifying software requirements for complex systems: New techniques and their application}, Software Engineering, IEEE Transactions on, 1980 (1): 2-13.

% \bibitem{nasa1}
% S. Easterbrook, J. Callahan, \emph{Formal methods for verification and validation of partial specifications: A case study}, Journal of Systems and Software, 1998, 40(3): 199-210.

% \bibitem{hangdian}
% S. Meyers, S. White, \emph{Software requirements methodology and tool study for A6-E technology transfer}, Grumman Aerospace Corp. , Report, 1983.

% \bibitem{timedCSP}
% G. M. Reed, A. W. Roscoe, \emph{A timed model for communicating sequential processes}, Automata, Languages and Programming, Springer Berlin Heidelberg, 1986: 314-323.

% \bibitem{CRSM}
% A. C. Shaw, \emph{Communicating real-time state machines}, Software Engineering, IEEE Transactions on, 1992, 18(9): 805-816.

% \bibitem{TA}
% R. Alur, D. L. Dill, \emph{A theory of timed automata}, Theoretical computer science, 1994, 126(2): 183-235.

% \bibitem{processcontrol}
% N. G. Leveson, M. P. E. Heimdahl, H. Hildreth, et al. \emph{Requirements specification for process-control systems}, Software Engineering, IEEE Transactions on, 1994, 20(9): 684-707.

% \bibitem{ChenSTeC}
% Y. Chen \emph{Stec: A location-triggered specification language for real-time systems}, Object/Component/Service-Oriented Real-Time Distributed Computing Workshops (ISORCW), 2012 15th IEEE International Symposium on. IEEE, 2012: 1-6.

% \bibitem{WuSTeC}
% H. Wu, Y. Chen and M. Zhang, \emph{On Denotational Semantics of Spatial-Temporal Consistency Language--STeC}, Theoretical Aspects of Software Engineering (TASE), 2013 International Symposium on. IEEE, 2013: 113-120.

% \bibitem{ChenSTeC2014}
% Y. Chen, Y. Zhang, \emph{A hybrid clock system related to STeC language}, Software Security and Reliability-Companion (SERE-C), 2014 IEEE Eighth International Conference on. IEEE, 2014: 199-203.

% \bibitem{ZhangSTeC}
% Y. Zhang, F. Mallet and Y. Chen, \emph{Timed Automata Semantics of Spatial-Temporal Consistency Language STeC}, Theoretical Aspects of Software Engineering Conference (TASE), 2014. IEEE, 2014: 201-208.

% \bibitem{observationmission1}
% N. Bianchessi, J. F. Cordeau, J. Desrosiers, et al., \emph{A heuristic for the multi-satellite, multi-orbit and multi-user management of earth observation satellites}, European Journal of Operational Research, 2007, 177(2): 750-762.

% \bibitem{maude}
% T. Luan, Y. Chen, J. Wang, \emph{Maude Rewriting System of Specification Language STeC for Real-time System}, Computer Engineering, 2013, 39(10): 57-62 (in Chinese).

% \bibitem{Uppaal}
% G. Behrmann, A. David, K. G. Larsen, \emph{A tutorial on uppaal}, Formal methods for the design of real-time systems. Springer Berlin Heidelberg, 2004: 200-236.


\end{thebibliography}

\chapter{总结与展望}

\section{论文总结}
软件应用领域广泛,软件可信综合考虑了软件的多个属性,是衡量软件质量的标准之一。全生命周期的软件可信评估对于认识当前软件的不足以及指导之后的软件开发过程有重要的参考意义。因此,本文面向轨道交通领域联锁软件的可信评估提出了相关模型与方法,具体包括下面三点:

首先,本文建立了轨道交通联锁软件可信量化评估模型。立足于四个可信属性,建立了可信评估指标体系;构建了轨道交通联锁软件可信评估的结构方程模型;根据结构方程模型的标准化结果,对因素负荷量进行归一化,计算度量元、子属性和属性的权重;结合可信度量模型的公式,自底向上计算子属性、属性、阶段以及软件整体的可信值;参考可信量化分级方法,判断当前软件可信值满足的条件,划分软件所属的可信等级。

其次,设计了两种可信性分配算法。对软件进行可信等级划分之后,如果软件的可信等级较低,需要提高软件的可信等级,即参考可信量化分级模型表,通过将子属性待提高的可信值分配给其下一级的度量元,实现软件整体可信值增加的目标。第一种分配算法是没有成本限制的条件下,按照度量元的改进优先指数从高到低进行分配,得出总改进成本;第二种分配算法是在受到成本约束条件下,基于优先选择单位贡献成本低的度量元进行分配的贪心选择策略,得出改进的最低成本。

最后,开发了轨道交通联锁软件可信评估工具。工具共包含三大模块,基本信息管理模块分两个子模块,分别为软件产品信息维护和软件人员信息维护;全生命周期可信评估模块分为需求阶段可信评估、设计阶段可信评估、编码阶段可信评估和测试阶段可信评估四个子模块,每个子模块有可信证据输入、权重分配和可信值计算三个功能;评估结果输出模块包括数据可视化和评估报告生成子模块,数据可视化主要是图表分析,用户可以直观各个阶段看到软件可信值分布,评估报告给出软件的可信等级以及改进意见。


\section{下一步工作}
本文的工作还有很多待完善之处,为了使软件评估结果更具有说服力,以下几点需要深入研究。

首先,软件全生命周期可以考虑增加使用阶段,研究使用阶段应该关注的属性、子属性和度量元具体涉及哪些方面,与前面四个阶段进行对照。可信评估指标体系需要进一步完善,本文选取了四个可信属性,适当地增加属性以及对应的二级评估指标。

其次,对可信值分配算法进行改进。将子属性可信值分配到度量元时,根据度量元所属定性和定量两种类型,分别处理。考虑实现成本一定的情况下,如何分配使软件的可信值达到最高。

最后,完善评估工具。从用户友好性出发,界面设计更加简洁,尽量减少用户手动操作;加强功能模块的封装以及接口的标准化等,以便提高工具的可扩展程度,并且降低二次开发的难度。




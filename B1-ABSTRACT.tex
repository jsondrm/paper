\vspace{-2.5cm}
\chapter*{\zihao{2}{摘~~~~要}}
\vspace{-1cm}

\setlength{\baselineskip}{25pt}
如今软件在各个领域应用非常广泛,但因为软件问题而引发的事故也频频出现。对于轨道交通领域来说,一旦发生事故,可能会引起交通瘫痪,严重甚至会威胁生命安全。通过对已完成的轨道交通联锁软件进行全生命周期的可信量化评估,可以更加全面地反映软件整体质量以及开发过程的完善性,以实现对未来软件开发流程的指导。实验室团队之前在航空航天领域和核装备领域开展了可信评估工作并应用于实际项目中,本文针对另一个安全攸关领域轨道交通进行研究。因此,本文工作围绕以下三个方面展开:

首先,建立了轨道交通联锁软件的可信量化评估模型。本文立足于四个可信属性,构建了面向全生命周期的可信量化评估指标体系和结构方程模型,据此计算度量元、子属性和属性的权重;结合可信度量模型的公式,自底向上计算子属性、属性、阶段以及软件整体的可信值;参考可信量化分级模型表,判断软件可信值满足的条件,划分软件的可信等级。

其次,基于可信量化分级模型表,设计了两种可信性分配算法。第一种算法按照度量元优先指数从高到低进行分配;第二种是基于贪心选择策略的分配算法,计算出使软件达到某一可信等级所需要的最低改进成本。

最后,开发了轨道交通联锁软件的可信量化评估工具。工具共包含三大模块:基本信息管理模块、全生命周期可信评估模块和评估结果输出模块。其中全生命周期可信评估模块分为需求、设计、编码和测试四个阶段可信评估子模块,每个子模块有可信证据输入、权重分配和可信值计算三个功能。该工具已应用于富欣智控公司研制的苏州有轨二号线联锁软件的可信评估。


% 如今软件在各个领域应用非常广泛,但由于软件问题而引发的事故也频频出现。软件可信性一直以来都是各行各界关注的话题,软件可信评估更是很多学者的研究课题。对于轨道交通领域来说,一旦发生事故,可能会引起交通瘫痪,严重甚至会威胁生命安全。保证轨道交通这种安全关键领域软件的高可信性意义重大。国家规定相关软件应用要严格按照国际标准的技术规范进行开发,遵守可靠性、可用性、可维护性和安全性的要求。通过对已完成的轨道交通联锁软件进行全生命周期的可信量化评估,可以更加全面地反映软件整体质量以及开发过程的完善性,以实现对未来软件开发流程的指导。实验室团队之前在航空航天领域和核装备领域开展了可信评估工作并应用于实际项目中,本文针对另一个安全攸关领域轨道交通进行研究。因此,本文工作围绕以下三个方面展开:

% \textbf{首先,建立了轨道交通联锁软件的可信量化评估模型。}本文立足于四个可信属性,构建了轨道交通联锁软件可信评估指标体系;在此基础上建立结构方程模型,根据结构方程模型路径图的标准化结果,对观测变量因素负荷量进行归一化,计算度量元、子属性和属性的权重;结合可信度量模型的公式,自底向上计算子属性、属性、阶段以及软件整体的可信值;参考可信量化分级模型表,判断软件可信值满足的条件,划分软件的可信等级。

% \textbf{其次,基于可信量化分级模型表,设计了两种可信性分配算法。}可信性分配是通过将子属性待提高的可信值分配给其下一级的度量元,实现软件整体可信值增加的目标。第一种分配算法是计算度量元的改进优先指数,按照指数由高至低进行分配,得出总改进成本;第二种是基于贪心选择策略的分配算法,计算出达到某一软件可信等级所需要的最低改进成本。

% \textbf{最后,开发了轨道交通联锁软件的可信量化评估工具。}工具共包含三大模块:基本信息管理模块、全生命周期可信评估模块和评估结果输出模块。其中全生命周期可信评估模块分为需求、设计、编码和测试四个阶段可信评估子模块,每个子模块有可信证据输入、权重分配和可信值计算三个功能;评估结果输出模块包括数据可视化和评估报告生成子模块,数据可视化主要是图表分析,评估报告给出软件的可信等级以及改进意见。该工具已应用于富欣智控公司研制的苏州有轨二号线联锁软件的可信评估。

\hspace{-0.5cm}
\sihao{\heiti{关键词:}} \xiaosi{可信软件,轨道交通,可信评估模型,可信性分配,量化评估工具}
